%!TeX root=../tese.tex
%("dica" para o editor de texto: este arquivo é parte de um documento maior)
% para saber mais: https://tex.stackexchange.com/q/78101

\chapter{CMS - White Label}
\label{chap:white-label}

\section{Definição e aplicações}
\subsection{White Label}

No contexto de mercado, a definição de White Label descreve uma terceirização do desenvolvimento de serviços, de produtos e, no caso desse projeto, de software. Em termos práticos, envolve um serviço base que posteriormente será personalizado e redistribuído para outras empresas com o possível propósito de poupar custos e recursos de produção. A maior vantagem que a prática oferece é flexibilizar a atribuição de responsabilidade do processo de criação e comercialização de um produto. A principal consequência é a economia de tempo, energia e dinheiro em termos de custos. Além disso, a definição não se restringe apenas a duas empresas. É possível, por exemplo, estabelecer três focos de responsabilidade –produção, marketing, vendas–, de forma que cada um seja atribuído a uma empresa distinta. Por exemplo, suponha que você possua uma loja de móveis artesanais de produção própria, o negócio vai bem, porém, a presença da sua marca na internet é nula e, sabendo disso, você pretende criar uma página para alavancar suas vendas criando uma loja virtual e aumentando a visualização do seu negócio com a divulgação, entretanto a falta de conhecimento no assunto e a mão de obra de alto custo se tornam impeditivos para continuar com a ideia, além disso, o tempo gasto na tarefa pode prejudicar a produção . É nesse contexto que se insere o White Label, ao invés de lidar com novas preocupações decide buscar no mercado alternativas e, dessa forma, acaba encontrando uma empresa terceirizada, especializada em desenvolvimento de sites. Ela lhe oferece uma estrutura para a página da sua loja, disponibilizando um modelo para que você insira sua marca da loja de móveis e possa realizar suas vendas. Este exemplo dá uma visão clara do que é um White Label pois estes softwares podem atender vastas necessidades como uma loja virtual, ou e-commerce, já que toda atualização técnica e a lógica de programação ficam a cargo da empresa desenvolvedora.

Historicamente, o conceito tem raízes na indústria musical e na produção em massa comum a partir da década de 1950. Com o intuito de evitar gastos em registros não suficientemente lucrativos, as produtoras forneciam cópias “em branco” a DJs reconhecidos que, por sua vez, as tocavam em eventos para analisar a recepção do público. Essa análise era decisiva na escolha dos registros que seriam lançados oficialmente. O sucesso desse método acarretou a popularização do White Label. Atualmente, as principais áreas que adotam a prática são as indústrias de vestuário, de produtos alimentícios genéricos e de tecnologia financeira (FinTech).

Para elucidar a presença do White Label nos diversos serviços e produtos da atualidade, no contexto de produtos genéricos alimentícios, a empresa WalMart e outros mercados adotam essa prática há tempos. É comum a presença de produtos personalizados com o logotipo do mercado onde são vendidos, todavia, esses estabelecimentos não são responsáveis pela manufatura do produto em questão. Além disso, no contexto de hardware, a Dell contrata a produção de seus monitores há anos, os quais são comercializados posteriormente sob propriedade dessa empresa. Por fim, as centrais de atendimento são majoritariamente conhecidas pela adoção do White Label.

Em suma, uma empresa é responsável pelo primeiro nível de atendimento ao cliente, de forma genérica, e no caso de maior complexidade, ocorre o encaminhamento às centrais especializadas da empresa proprietária.

Para as empresas contratadas, a maior facilidade adquirida envolve a expansão da oferta visto que é possível contornar o impacto do ingresso no mercado no qual marcas maiores e antigas exercem dominância. Em outras palavras, além de amenizar o risco de reprovação do produto, é possível fortalecer a presença da marca no mercado. Em contrapartida, a empresa proprietária poupa custos de produção e, por conseguinte, pode conduzir mais investimentos para o crescimento da empresa através do marketing e otimização de vendas. 
Dado que a solidificação das marcas deve-se à presença no mercado quantitativamente e ao alcance do público alvo, o White Label permite tanto a oferta de produtos em escala, como o enfoque na divulgação e crescimento da empresa. Abaixo, segue exemplos de marcas e produtos que utilizam do white label no contexto de software:

\begin{itemize}
    \item {\bfseries Shopify}: Uma empresa que desenvolve software para lojas online, nela é possível criar e gerenciar uma loja virtual. A ideia foi baseada em um software anterior desenvolvido pelos próprios fundadores da Shopify que servia como loja virtual de snowboard da qual eles eram proprietários.

    \item {\bfseries Zendesk}: Empresa de desenvolvimento de software que oferece uma plataforma para o serviço de atendimento ao cliente através de tickets. Há também o oferecimento personalizado destes pela empresa, de forma que a empresa contratante pode ter o seu próprio atendimento por tickets seguindo assim a ideia do white label.

    \item {\bfseries WordPress}: Embora não seja um software em si, o site oferece um CMS, que está em um núcleo de um software white label, que permite usuários criarem seu próprio site ou blog da forma que mais os agrada sem se preocupar com manutenção de código e outros cuidados exigidos por uma aplicação web.
\end{itemize}

\subsection{Content Management System (CMS)}
Um Sistema Gerenciador de Conteúdo (CMS) é uma aplicação usada, como diz o próprio nome, para gerenciar conteúdo, que permite criar, publicar e editar sem ter o conhecimento de linguagens de programação e a linguagem de marcação, HTML, em outras palavras, se você quisesse criar uma página do zero, como um blog, seria necessário ter um conhecimento básico em HTML, com o auxílio do CMS é possível criá-lo sem tais preocupações, pois, com o WordPress, um dos gerenciadores de conteúdo já citados previamente, é possível desenvolver sua página apenas com as ferramentas intuitivas disponibilizadas por ele, escolhendo estilo para fontes, posicionamento dos elementos, inserção de imagens e diversas outras opções. Diversos elementos de uma página online podem ser interpretados como o conteúdo, por exemplo, um simples texto, uma imagem, um vídeo ou até mesmo cores que a página possui.

O núcleo de um CMS pode ser dividido em duas partes:
\begin{itemize}
    \item {\bfseries Content Management Application (CMA)}: O aplicativo que permite a inserção dos elementos da página, como texto, imagens, etc.

    \item {\bfseries Content Delivery Application (CDA)}: O backend, que trata os elementos que você escolheu inserir na sua página web e os torna visiveis de forma responsiva ao visitante.
\end{itemize}

\section{No Software e no Projeto}

No cenário da Tecnologia, os softwares white-label frequentemente são comercializados sob o modelo Software as a Service (SaaS). O SaaS é um modelo de entrega de software baseado na nuvem, onde os aplicativos são acessados por meio da internet. A principal característica dessa abordagem é a flexibilidade e acessibilidade global para os usuários, com atualizações automáticas e a capacidade de acessar aplicativos de qualquer lugar. No contexto do white label, o SaaS oferece às empresas a oportunidade de personalizar e comercializar serviços de software sob sua própria marca. Isso significa que as empresas podem adaptar completamente a aparência e identidade visual do aplicativo SaaS, incluindo o uso de seus próprios logotipos e esquemas de cores, criando uma experiência de marca consistente.

Além disso, as empresas podem personalizar os recursos do aplicativo SaaS para atender a determinadas necessidades e as de seus clientes. Isso envolve a adaptação de funcionalidades específicas e a adição de recursos exclusivos que são relevantes para seu mercado ou setor. A grande vantagem é que as empresas que adotam uma solução SaaS white label podem revender ou distribuir o software sob sua própria marca, criando uma nova fonte de receita. Isso é particularmente útil para empresas que desejam expandir seus serviços sem gastar recursos no desenvolvimento de software personalizado.

Essa abordagem também permite que as empresas se concentrem no atendimento às necessidades de seus clientes finais, fornecendo uma solução de software personalizada que atenda às demandas do mercado, em vez de gastar recursos no desenvolvimento de software desde o início. Como resultado, a relação entre SaaS e white label é altamente sinérgica, oferecendo flexibilidade, escalabilidade e oportunidades de negócios adicionais para empresas que desejam personalizar e comercializar serviços de software.

O conceito de "white label" na indústria de tecnologia é uma estratégia que permite a empresas desenvolver produtos, serviços ou soluções e, em seguida, licenciar ou fornecer esses produtos para outras empresas, que têm a liberdade de personalizá-los e comercializá-los sob suas próprias marcas. Essa abordagem é valiosa para empresas que desejam ampliar sua oferta de tecnologia sem o ônus de desenvolver tudo internamente. 

Dessa forma, empresas de tecnologia muitas vezes disponibilizam software white label que pode ser adaptado e rebrandizado por outras empresas. Isso é particularmente evidente em setores como marketing digital, onde agências podem usar uma plataforma de automação de marketing white label e adaptá-la de acordo com as necessidades de seus clientes. Além do software, as plataformas white label são outro exemplo notável. Essas plataformas fornecem uma base sólida sobre a qual outras empresas podem construir produtos ou serviços personalizados. Por exemplo, uma empresa pode aproveitar uma plataforma de comércio eletrônico white label para criar uma loja online exclusiva para seus clientes, economizando tempo e recursos de desenvolvimento. 

Outra abordagem inclui empresas que fabricam hardware, como smartphones, que também adotam o white label. Elas permitem que outras empresas personalizem e comercializem esses dispositivos sob suas próprias marcas, aproveitando a infraestrutura já existente e adicionando valor aos produtos. Além disso, o white label é aplicado em setores de pagamento, segurança cibernética e serviços em nuvem, proporcionando flexibilidade e escalabilidade às empresas que desejam expandir suas ofertas tecnológicas sem o esforço de criação do zero. Empresas como Shopify, WordPress e Google Workspace são exemplos notáveis de sucesso na implementação de modelos white label, demonstrando a eficácia dessa estratégia na indústria de tecnologia. 

Em contrapartida, ao adotar produtos ou serviços white label na indústria de tecnologia, as empresas podem se deparar com desafios que requerem atenção cuidadosa. Um desses desafios é a dependência de fornecedores. Quando uma empresa escolhe essa abordagem, ela se torna intimamente ligada ao fornecedor original para manter e atualizar o produto ou serviço. Problemas financeiros, alterações nos termos contratuais ou a suspensão do suporte por parte do fornecedor podem afetar negativamente as operações da empresa cliente e a qualidade dos serviços oferecidos aos seus próprios clientes. Além disso, as limitações de personalização podem se mostrar um desafio. 

Embora o white label ofereça flexibilidade para personalizar produtos ou serviços, existem restrições. Empresas podem se deparar com limitações na personalização de recursos essenciais, integração com outros sistemas ou adaptação completa à marca. Isso pode resultar em soluções que não atendem completamente às necessidades exclusivas da empresa cliente ou de seus clientes finais. Outro desafio é a competição e diferenciação. Quando várias empresas utilizam a mesma solução white label e realizam apenas personalizações superficiais, pode ser difícil para cada empresa se destacar e oferecer um valor exclusivo aos clientes, o que pode afetar sua vantagem competitiva. 

Além disso, o controle limitado sobre atualizações e desenvolvimento é uma preocupação. A empresa cliente pode não ter controle total sobre o cronograma de atualizações e desenvolvimento do produto ou serviço white label. Isso pode afetar a capacidade da empresa de responder rapidamente às necessidades do mercado ou de seus clientes, especialmente se o fornecedor não estiver alinhado com suas prioridades. Há também riscos associados à qualidade e segurança. A qualidade e a segurança dos produtos ou serviços white label dependem da competência e integridade do fornecedor original. Se o fornecedor não conseguir manter os padrões de qualidade ou segurança esperados, isso pode prejudicar a reputação da empresa cliente e a confiança de seus clientes.

Por fim, custos ocultos podem surgir ao longo do tempo, apesar da economia inicial proporcionada pelo white label. Isso pode incluir taxas de licenciamento, custos de suporte ou custos de integração com outros sistemas, que podem impactar o orçamento da empresa cliente. Para lidar com esses desafios, é fundamental realizar uma análise minuciosa dos fornecedores, contratos e acordos de suporte. Além disso, é importante manter a flexibilidade e estar preparado para ajustar estratégias conforme necessário, como a possibilidade de mudar de fornecedor ou desenvolver recursos personalizados adicionais para atender às necessidades específicas dos clientes.

Tendo em vista os problemas relacionados a implementação do White Label, é possível contornar cada problema da forma mais eficaz encontrada. No caso do projeto analisado, CPqs Abertas, foi utilizada uma plataforma que permitia grande parte das personalizações que o White Label dificulta. Cada plataforma web criada possui uma instância no Strapi associada, além de componentes já criados pela equipe de desenvolvimento. Vale constar que é possível criar inúmeros componentes, de forma que cada um será incluido no design da instância de interesse. 

Com o intuito de garantir designs compatíveis com a melhor experiência de usuário, a equipe de desenvolvimento trabalhou em conjunto com uma equipe de design. Dessa forma, foi possível discutir vias relacionadas a responsividade, usabilidade, tipografia, estilo visual individual para cada unidade, além das especificações únicas que cada uma apresentava. 

Em suma, o White Label viabilizou a manutenção dos elementos de identidade visual do projeto de forma rápida e eficiente. Em contrapartida, a equipe agregou ferramentas para viabilizar simultaneamente a personalização de cada plataforma. Ademais, em relação a custos e dependência de fornecedores, que constituem problemas comuns na abordagem white label, não há grandes preocupações visto que o projeto apresenta uma visão voltada ao código aberto e contribuição. 

\section{Benefícios do CMS}

O desenvolvimento eficiente e eficaz de sites é um desafio central para empresas que desejam estabelecer uma presença sólida na web. Uma abordagem promissora para enfrentar esse desafio é a utilização de uma base de CMS (Sistema de Gerenciamento de Conteúdo) compartilhada para diversos sites. Nesta seção, serão exploradas as vantagens dessa estratégia, que incluem a economia de tempo e recursos, a padronização, a personalização flexível e a economia de custos. Outro ponto abordado serão os benefícios de atualizações simplificadas, gestão de conteúdo centralizada, escalabilidade e suporte técnico. Além disso, será discutido como essa abordagem pode manter a consistência da marca e acelerar o tempo de lançamento, ao mesmo tempo em que oferece flexibilidade para atender às necessidades individuais de cada projeto.

A eficiência de desenvolvimento é uma das vantagens notáveis ao utilizar a mesma estrutura e funcionalidade do CMS (Sistema de Gerenciamento de Conteúdo) para vários sites. Essa abordagem economiza considerável tempo e esforço de desenvolvimento, eliminando a necessidade de criar um CMS personalizado para cada projeto. Dessa forma, é possível direcionar os recursos de forma mais eficaz, com enfoque em aprimorar e personalizar as partes específicas de cada site, ao invés de reinventar a roda no desenvolvimento do sistema de gerenciamento de conteúdo. Isso não apenas acelera a implementação de projetos, mas também promove uma abordagem mais consistente e eficaz para o gerenciamento de conteúdo em várias propriedades da web.

Além disso, a padronização desempenha um papel crucial ao empregar a mesma base de CMS em todos os sites. Isso garante um alto padrão de qualidade, segurança e desempenho em todas as implementações, resultando em uma experiência consistente para os usuários. Essa consistência não apenas aumenta a confiança do usuário, mas também simplifica a manutenção e o gerenciamento contínuo. Essa uniformidade na qualidade é fundamental para estabelecer uma presença online sólida e confiável.

Outro aspecto importante é a economia de custos que desempenha um papel significativo. Essa economia é substancial, uma vez que elimina a necessidade de desenvolver e manter um CMS personalizado para cada site. Além disso, o uso de uma solução white label pode frequentemente ser mais acessível em comparação com o desenvolvimento interno de um CMS completo, que envolve custos significativos de desenvolvimento, manutenção e atualizações contínuas. Dessa forma, as empresas podem otimizar seus recursos financeiros, direcionando-os para outras áreas críticas de seus projetos, enquanto ainda desfrutam de uma infraestrutura robusta de gerenciamento de conteúdo. Isso resulta em uma gestão financeira mais eficiente e um ROI (Retorno sobre Investimento) mais favorável para as empresas.

Em relação à manutenção, atualizações simplificadas permitem que as atualizações de segurança, correções de bugs e melhorias de recursos sejam aplicadas de forma mais eficiente, uma vez que todos os sites estão construídos sobre uma única base CMS compartilhada. Isso simplifica o processo de manutenção, pois as atualizações podem ser aplicadas de forma centralizada, garantindo que todos os sites se beneficiem das últimas melhorias sem a necessidade de lidar com atualizações individuais para cada site. Isso não apenas economiza tempo, mas também fortalece a segurança e o desempenho geral, proporcionando uma gestão mais eficaz dos sites da empresa. No geral, a gestão de conteúdo centralizada é uma das vantagens-chave , visto que o gerenciamento de conteúdo para todos os sites pode ser conduzido a partir de um único local, simplificando consideravelmente o processo de atualização e publicação de conteúdo. Isso significa que as equipes de conteúdo podem gerenciar, editar e publicar informações em todos os sites de forma eficiente e consistente, sem a necessidade de acessar sistemas separados para cada propriedade da web. Essa centralização do controle facilita a manutenção da precisão e da coesão do conteúdo em todos os sites, proporcionando uma gestão de conteúdo mais eficaz e economia de tempo para a empresa.

A escalabilidade é uma das principais vantagens quando se utiliza uma base de CMS compartilhada para vários sites, haja vista que permite que você expanda facilmente o número de sites que utilizam a mesma base CMS à medida que o seu negócio cresce, tudo isso sem a necessidade de recriar a infraestrutura do CMS para cada novo site. Isso significa que você pode adicionar novos sites à sua rede online de forma ágil e eficiente, aproveitando a estrutura e funcionalidade já estabelecidas. Essa flexibilidade é crucial para plataformas em crescimento, pois permite acompanhar a expansão do seu portfólio de sites de maneira eficaz, economizando tempo e recursos que, de outra forma, seriam gastos na criação de CMS individuais para cada novo projeto.

Mesmo no contexto de escalabilidade, a consistência da marca é mantida. É possível manter a uniformidade na identidade visual e elementos-chave de design em todos os projetos. Ao compartilhar a mesma base de CMS, as plataformas podem garantir que a marca seja representada de maneira consistente em todos os sites, mantendo logotipos, cores, tipografia e outros elementos de design alinhados. Essa uniformidade fortalece a identidade da marca, cria confiança nos visitantes e oferece uma experiência de usuário coesa em todas as propriedades da web. Além disso, simplifica o processo de atualização de elementos de marca em todos os sites, garantindo que qualquer alteração seja refletida de forma consistente em toda a presença online da empresa.

Uma das vantagens fundamentais do uso de um CMS no contexto do Cpqsabertas é o fato de ser “user-friendly”, isto é, de fácil entendimento e intuitivo, tendo em vista que estamos constantemente em contato com designers, eles, como usuários do gerenciador de conteúdo, não precisam entender profundamente de código e linguagens de programação para contribuir com alterações como cores, escolhas de logos, textos e estilos no geral. O código que está por trás do CMS já cuida disso ao preencher os campos criados pelos desenvolvedores.

Por fim, garante-se a rapidez no tempo de lançamento. Com a infraestrutura do CMS construída, é possível lançar novos sites de forma significativamente mais rápida. Isso resulta em economia de tempo substancial e permite que seus clientes entrem no mercado mais cedo, ganhando uma vantagem competitiva valiosa. 

No entanto, é importante estar ciente de que a flexibilidade pode ser limitada ao adotar essa abordagem. Para abordar essa desvantagem, uma possível solução é implementar módulos de personalização que permitam a adaptação de recursos específicos para atender às necessidades individuais de cada projeto, mantendo ao mesmo tempo a eficiência geral do CMS compartilhado. Dessa forma, pode-se equilibrar a rapidez no lançamento com a capacidade de personalizar e atender às demandas exclusivas de seus clientes.