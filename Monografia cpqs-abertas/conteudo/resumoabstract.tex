%!TeX root=../tese.tex
%("dica" para o editor de texto: este arquivo é parte de um documento maior)
% para saber mais: https://tex.stackexchange.com/q/78101

% As palavras-chave são obrigatórias, em português e em inglês, e devem ser
% definidas antes do resumo/abstract. Acrescente quantas forem necessárias.
\palavrachave{CPqs Abertas}
\palavrachave{CMS}
\palavrachave{White Label}
\palavrachave{Arquitetura de Software}

\keyword{CPqs Abertas}
\keyword{CMS}
\keyword{White Label}
\keyword{Software Architecture}

% O resumo é obrigatório, em português e inglês. Estes comandos também
% geram automaticamente a referência para o próprio documento, conforme
% as normas sugeridas da USP.
\resumo{
Esta monografia visa expor a evolução e o desenvolvimento do projeto CPqs Abertas ao longo do ano de 2023, apresentando as atividades desenvolvidas, tal como melhorias implementadas, inserção de novas tecnologias e modificação da arquitetura. O trabalho efetuado no decorrer dos dois semestres teve como objetivo mudar a arquitetura para tornar a inserção de institutos algo escalável e prático por meio da estratégia White Label + CMS utilizando a ferramenta Strapi, dessa forma, o maior esforço durante o primeiro ciclo, foi a adaptação do código para encaixarmos a estratégia citada acima no projeto, com a inserção dela conseguimos unificar o código dos institutos já existentes, sendo essa uma das melhorias implementadas pela equipe, uma vez que, esse avanço permite a redução exponencial de tempo usado para a inserção de um novo instituto além de evitar erros uma vez que, da forma que nos foi apresentado o projeto possuía muito código replicado. Além disso, adicionamos ao projeto CPqs Abertas o IFUSP, cuja inserção nos garantiu a escalabilidade desejada desde o início dos trabalhos. Este avanço não só permitirá um trabalho mais ágil para futuros desenvolvedores na inserção de novas unidades mas também facilita a expansão do projeto como um todo. 
}

\abstract{
This paper aims to expose the evolution and development of the CPqs Abertas project throughout the year of 2023, presenting the activities carried out, such as, implemented improvements, the integration of new technologies, and modification of the architecture. The work conducted over those two semesters aimed to change the architecture to make the integration of institutes scalable and practical through the White Label + CMS strategy, using the Strapi tool. Thus, the major effort during the first cycle was adapting the code to fit the mentioned strategy into the project. With its integration, we were able to unify the code of existing institutes, marking it as one of the improvements implemented by the team. This advancement allows for an exponential reduction in the time used to integrate a new institute and helps prevent errors, considering that the project, as initially presented, had a significant amount of duplicated code. Additionally, we added IFUSP to the CPqs Abertas project, whose integration ensured the scalability desired from the beginning of the project. This progress not only enables more efficient work for future developers in integrating new units but also makes it easier the overall expansion of the project.
}
