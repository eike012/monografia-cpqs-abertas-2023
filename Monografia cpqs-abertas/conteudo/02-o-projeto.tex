%!TeX root=../tese.tex
%("dica" para o editor de texto: este arquivo é parte de um documento maior)
% para saber mais: https://tex.stackexchange.com/q/78101


% https://docs.google.com/document/d/1uTbyVusuIyXaw3bWyG_wQxeqQG_cjSS4FdGDaHG2rUw/edit LINK DO ESBOÇO NO WORD

\chapter{O Projeto Cpqs Abertas}
\label{chap:o-projeto}

O Projeto Cpqs-abertas, inicialmente denominado FAUaberta, teve origem como parte integrante da disciplina MAC 5716 - Laboratório de Programação Extrema, ministrada pelo Instituto de Matemática e Estatística (IME) da Universidade de São Paulo (USP). Sua concepção emergiu do desejo de ampliar a divulgação da produção intelectual da Faculdade de Arquitetura e Urbanismo (FAU), tornando-a mais acessível e visível não apenas à comunidade USP, mas também a acadêmicos em todo o Brasil. Essa iniciativa estava fundamentada no compromisso inabalável das universidades públicas de disseminar suas pesquisas e descobertas, destacando claramente seus impactos nos âmbitos social, econômico e tecnológico.

Na fase inicial do Projeto Cpqs-abertas, anteriormente denominado FAUaberta, um componente essencial foi a extração de informações dos currículos Lattes dos docentes da FAU. Isso resultou na criação de um repositório rico em informações acadêmicas, tornando-se um dos pilares do projeto. No entanto, a verdadeira essência do sucesso do projeto reside na colaboração ativa e incansável de uma equipe diversificada, composta por estudantes e professores de diversos institutos da USP. Esse espírito colaborativo não apenas promoveu a coleta de dados de alta qualidade, mas também estabeleceu as bases para a evolução subsequente do projeto.

O ano de 2020 marcou um ponto crucial na história do projeto, com o lançamento da página FAUaberta. Nesse ponto, uma decisão estratégica fundamental foi tomada: a expansão do projeto para abranger outras unidades da USP. Dessa forma, o Cpqs-abertas tomou forma, incorporando não apenas a FAU, mas também o Instituto de Matemática e Estatística (IME) e a Faculdade de Economia, Administração e Contabilidade de Ribeirão Preto (FEARP) da USP. Esse movimento estratégico demonstrou a adaptabilidade do projeto, abrindo as portas para um crescimento contínuo e a inclusão de mais unidades da universidade.

No entanto, com a chegada de novas unidades, surgiram desafios de outra natureza. O principal deles era a presença de códigos replicados. Cada instituto mantinha seu próprio código-fonte, que, embora semelhante aos das outras unidades, resultava em duplicação de esforços e dificuldades na manutenção. Para contornar esse obstáculo, o Cpqs-abertas adotou a estratégia de "white label", conceito que será explicado nas seções a seguir, e fez uso de um Sistema de Gerenciamento de Conteúdo (CMS). Essa abordagem revolucionária permitiu que as unidades compartilhassem o mesmo código, simplificando significativamente a adição de novas unidades e reduzindo o tempo necessário para desenvolver uma nova página, que agora mantinha uma consistência de design praticamente idêntica às demais unidades.

A estratégia de "white label" e o uso de um Sistema de Gerenciamento de Conteúdo (CMS) revelaram-se pilares cruciais na evolução e eficiência do projeto Cpqs-abertas. Esses elementos técnicos desempenham um papel fundamental na unificação e manutenção contínua do projeto, permitindo que ele cumpra sua missão de disseminar conhecimento de forma eficaz e colaborativa.

Ao adotar a abordagem "white label", o projeto conseguiu eliminar uma das barreiras mais significativas que surgem com a expansão para novas unidades. Tradicionalmente, cada instituto mantinha seu próprio código-fonte, resultando em complexidade desnecessária e duplicação de esforços. A implementação da estratégia "white label" permitiu que todas as unidades compartilhassem uma base de código comum. Isso não apenas simplificou a adição de novas unidades à plataforma, mas também tornou mais eficiente a manutenção contínua. As atualizações e melhorias podem agora ser aplicadas de forma consistente em todas as páginas do projeto, garantindo uma experiência uniforme para os usuários.

O papel central dos CMS na gestão do conteúdo do projeto não pode ser subestimado. Essas plataformas fornecem uma infraestrutura flexível que permite aos administradores e colaboradores adicionar, editar e organizar o conteúdo de maneira eficiente. Isso é especialmente valioso em um projeto de grande escala como o Cpqs-abertas, onde uma vasta quantidade de informações acadêmicas está sendo gerenciada e atualizada constantemente.

A capacidade de personalizar e adaptar facilmente as páginas das diferentes unidades do projeto é um dos benefícios mais notáveis dos CMS. Cada unidade pode manter sua identidade visual única, mas ao mesmo tempo aproveitar as melhorias globais implementadas no projeto. Isso garante que a plataforma seja coesa e amigável, ao mesmo tempo que permite a expressão da individualidade de cada unidade da USP.

Além disso, os CMS proporcionam uma base sólida para a integração de recursos avançados, como mecanismos de busca poderosos, análises de dados e recursos interativos. Isso torna o Cpqs-abertas não apenas uma fonte de informação estática, mas também uma ferramenta dinâmica para a pesquisa acadêmica, facilitando a descoberta de conteúdo relevante e a colaboração entre acadêmicos.

À medida que o Projeto Cpqs-abertas continua a expandir sua presença e atrair mais unidades da USP, torna-se cada vez mais evidente a importância dos CMS e da estratégia "white label". Esses elementos técnicos desempenham um papel crucial ao proporcionar a flexibilidade e a escalabilidade necessárias para sustentar uma plataforma de disseminação de conhecimento acadêmico em constante crescimento. A próxima seção da monografia explora a infraestrutura tecnológica subjacente em maior profundidade, revelando como esses componentes técnicos operam em conjunto para tornar o Cpqs-abertas uma realidade viável e eficaz.